
\documentclass[10pt,a4paper]{article}
\usepackage[utf8]{inputenc}
\usepackage[russian,english]{babel}
\usepackage[T2A]{fontenc}
\usepackage{amsmath}
\usepackage{amsfonts}
\usepackage{amssymb}
\title{Инвариантная самостоятельная работа 5

Команды для создания текстового документа}
\author{Иванов Иван}
\date{18 февраля 2022}
\begin{document}
\maketitle
\newpage

pagestyle\{empty\} - стиль страницы без номера

pagestyle\{myheadings\} — стиль страницы с номером в правом
верхнем углу

setcounter\{page\}\{3\} — начать нумерацию страниц с 3

\section{Выравнивание — меню Insert–Environments}

Выравнивание по центру - begin\{center\} Текст end\{center\}

Выравнивание по правому краю - begin\{flushright\} Текст end\{flushright\}

Выравнивание по левому краю - begin\{flushleft\} Текст end\{flushleft\}

vspace\{5mm\} — мягкий вертикальный отступ (в некоторых случаях
игнорируется, например, если попадает на начало страницы)

vspace*\{10mm\} — вертикальный отступ

hspace\{20mm\} — мягкий горизонтальный отступ (в некоторых
случаях игнорируется, например, если попадает на конец строки).

hspace*\{20mm\} — Горизонтальный отступ.

newline — переход на новую строку внутри одного абзаца,
предыдущая строка не растягивается.

linebreak — переход на новую строку внутри абзаца, предыдущая
строка растягивается по ширине.

pagebreak — переход на новую страницу, предыдущая растягивается
newpage — переход на новую страницу

\section{Размеры шрифта}

footnotesize — \footnotesize Размер шрифта \normalsize

small — \small Размер шрифта \normalsize

normalsize — \normalsize Размер основного текста

large — \large Размер шрифта \normalsize

Large — \large Размер шрифта \normalsize

LARGE — \large Размер шрифта \normalsize

huge — \huge Размер шрифта \normalsize

Huge — \huge Размер шрифта \normalsize

\section{Начертание шрифта}

Панель инструментов Typeface.

\{bf Текст \} или textbf\{Bold\} (меню Typeface) — {\bf полужирный текст}


\{it Текст\} или textit\{italic\} (меню Typeface) — {\it курсив}

textbf\{textit\{Текст\}\} — \textbf{\textit {полужирный курсив}}

underline\{Текст\} (меню Math) — \underline{подчеркивание}

Линии разной толщины rule\{35mm\}\{.3pt\} и rule\{35mm\}\{.10pt\}

\rule{35mm}{.3pt}

\rule{35mm}{.10pt}

tableofcontents — создает содержание
\end{document}
