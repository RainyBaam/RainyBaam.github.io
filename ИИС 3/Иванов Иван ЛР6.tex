% "Лабораторная работа 2"

\documentclass[a4paper,12pt]{article} % тип документа

% report, book

%  Русский язык

\usepackage[T2A]{fontenc}			% кодировка
\usepackage[utf8]{inputenc}			% кодировка исходного текста
\usepackage[english,russian]{babel}	% локализация и переносы


% Математика
\usepackage{amsmath,amsfonts,amssymb,amsthm,mathtools} 


\usepackage{wasysym}


%Заговолок
\author{Иванов Иван ИВТ 3 курс}
\title{Особености технологии набора технического текста в \LaTeX{}}
\date{\today}


\begin{document} % начало документа
\maketitle
\newpage

\begin{center}
\section{Для чего предназначена издательская система LATEX?}
\end{center}

\huge TEX \normalsize — это издательская система, предназначенная для набора
научно-технических текстов высокого полиграфического
качества. LATEX — один из наиболее популярных макропакетов на базе
TEXа, существенно дополняющий его возможности. Создаваемые с
помощью LATEXа тексты могут содержать математические формулы,
таблицы и графические изображения. Поддерживается автоматическая
нумерация страниц, разделов, формул и пунктов перечней. Система
сама генерирует оглавление, списки таблиц и иллюстраций,
перекрёстные ссылки, сноски, колонтитулы и предметный указатель.
Наконец, имеется возможность определять собственные макрокоманды
и стили.


\section{В каких случаях рационально использовать LATEX?}

LATEX позволяет писать хорошо структурированные документы. Но в
нем сложно и долго создавать полностью новый макет, следовательно
трудно создать небольшие документы. Может использоваться для верстки академических, летературных текстов и нотных грамот.

\section{Какие преимущества имеет работа в этот системе?}

Преимущества LaTeX для академического использования состоят в том,
что он производит разумно свёрстанные документы, которые хорошо выглядят именно в таком виде, в каком представители научных кругов
обычно любят публиковать документы.

\section{Какие сложности могут возникнуть при работе в этой системе?}

Сверстать документ так, чтобы его было приятно и удобно читать – это
далеко не такая простая задача. Неопытному пользователю может показаться слишком громоздким набор из множества команд.

\section{Какие недостатки отмечают пользователи при работе с этой системой?}

\begin{itemize}
\item Готовый результат можно увидеть только после сборки.

\item Набранный текст в LaTeX есть полноценный программный код. Во
время обучения будет очень тяжело найти ошибку.

\item Наличие большого количества не очевидных случаев, которые решаются с помощью гугла.

\item Требуется потратить от недели до нескольких месяцев на обучение
\end{itemize}
\end{document} % конец документа