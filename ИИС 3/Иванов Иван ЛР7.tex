% "Лабораторная работа 2"

\documentclass[a4paper,12pt]{article} % тип документа

% report, book

%  Русский язык

\usepackage[T2A]{fontenc}			% кодировка
\usepackage[utf8]{inputenc}			% кодировка исходного текста
\usepackage[english,russian]{babel}	% локализация и переносы


% Математика
\usepackage{amsmath,amsfonts,amssymb,amsthm,mathtools} 


\usepackage{wasysym}


%Заговолок
\author{Иванов Иван 3ИВТ1 ИИТиТО}
\title{Лабораторная работа 7\\Особенности технологии создания текста с формулами в \LaTeX{}}
\date{\today}


\begin{document} % начало документа
\maketitle
\newpage

\section{Формулы}
\subsection{Встраиваемая (включенная) формула)}
Площадь прямоугольника определяется по формуле $S=ab$ известной из школьного курса математики. Например, $ 2+2=4 $ называют равенством.

\section{Выключенная формула}
Формула по центру строки \[1+3=4\]
Теорема Пифагора $$ a^2+b^2=c^2 $$ часто применяется при решении различных геометрических задач.

\subsection{Нумерация формул}

\begin{equation}
a+b=b+a
\end{equation}

\begin{equation} \label{pifagor}
a^2+b^2=c^2
\end{equation}

Чтобы ссылаться на формулу, которая стоит в тексте много раньше, можно использовать команду \textbf{eqref}\\
Например.\\
Как было сказано раньше в
\eqref{pifagor} гипотенуза определена.
Об этом было уже сказано на странице \pageref{pifagor}

\section {Дроби}

$\frac{1}{4}+\frac{1}{4}= \frac{1}{2}$ это больше по высоте, чем текст. Чтобы не изменять внешний вид текста используют выключенные формулы.\\

Поэтому в случае использования обыкновенных дробей используйте выключенные формулы.
\[\frac{1}{4}+\frac{1}{4}= \frac{1}{2}\]
\newpage

\section {Скобки}

\[(2+3)*5=25 \]
\[(2+3)\times 5=25\]
\[(2+3) \cdot 5=25\]
\subsection*{Размер скобок}
\[(\frac{4}{2}+3) \cdot 5=25\]

\[\left(\frac{4}{2}+3\right) \cdot 5=25\] Размер подбирается автоматически для любых скобок при использовании \textbf{left и right}

\[\{2+3\} \cdot 5=25\]

\section {Индексы и показатели}
$m_1$\\
$c^2$

Если аргумент состоит из более чем одного символа, то его следует взять в фигурные скобки.\\
$m_{12}$\\
$c^{22}$\\

\section {Стандартные функции}
$\sin x=0\\
\arctg x=\sqrt{3}\\
\arcctg a=\sqrt[5]{3}\\
\log_{x-1}{(x^2+3x-4)}\geqslant2\\
\lg x=\ln a $\\
$$ \sum_{i=1}^{n}a_i+b_j $$ \\
\[\sum_{i=1}^{n}a_i+b_j\]
\begin{equation}
\sum_{i=1}^{n}a_i+b_j
\end{equation}

\subsection*{Интеграл}
$ l=\int r^2dm\\
l=\int_{0}^{1}r^2dm $ \\
\[l=\int\limits_{0}^{1}r^2dm\]


\section*{Задание 2}
\[\int\frac{dx}{\ln x} =\ln|\ln x|+ \sum_{i=1}^{\infty}\frac{(\ln x)^i}{i \cdot i!}\]

\[\int\frac{dx}{(\ln x)^n}=-\frac{x}{(n-1)(\ln x)^{n-1}}+\frac{1}{n-1}\int\frac{dx}{(\ln x)^{n-1}}\]

\[\int x^m\ln x dx=x^{m+1}\left(\frac{\ln x}{m+1}-\frac{1}{(m+1)^2}\right)\]

\[\int x^m(\ln x)^n dx = \frac{x^{m+1}(\ln x)^n}{m+1} - \frac{n}{m+1}\int x^m(\ln x)^{n-1}dx\]

\[\int\frac{(\ln x)^n dx}{x}=\frac{(\ln x)^{n+1}}{n+1}\]

\[\int\frac{\ln x dx}{x^m}=-\frac{\ln x}{(m-1)x^{m-1}}-\frac{1}{(m-1)^2x^{m-1}}\]

\[\int \frac{(\ln x)^n dx}{x^m}=-\frac{(\ln x)^n}{(m-1)x^{m-1}}+\frac{n}{m-1}\int\frac{(\ln x)^{n-1}dx}{x^m}\]

\end{document} % конец документа