% "Лабораторная работа 2"

\documentclass[a4paper,12pt]{article} % тип документа

% report, book

%  Русский язык

\usepackage[T2A]{fontenc}			% кодировка
\usepackage[utf8]{inputenc}			% кодировка исходного текста
\usepackage[english,russian]{babel}	% локализация и переносы


% Математика
\usepackage{amsmath,amsfonts,amssymb,amsthm,mathtools} 


\usepackage{wasysym}


%Заговолок
\author{Иванов Иван ИВТ 3 курс}
\title{Лабораторная работа 9\\
Создание таблицы средствами \LaTeX{}}
\date{\today}


\begin{document} % начало документа
\maketitle
\newpage

\section{Создание таблицы}
\subsection{Инструменты}
\begin{enumerate}
\item Вкладка Помощник
\item Быстрая таблица
\end{enumerate}
\subsection{Шаг 1}
Выберите количество столбцов и количество строк\\
\begin{tabular}{|c|c|c|c|c|}
\hline
- & - & - & - & - \\
\hline
- & - & - & - & - \\
\hline
- & - & - & - & - \\
\hline
- & - & - & - & - \\
\hline
- & - & - & - & - \\
\hline
\end{tabular}

\subsection{Шаг 2}
Введите текст в таблицу\\
\begin{tabular}{|c|c|c|c|c|}
\hline
№&Критерии&Параметры&-&-\\
\hline
-&-&выполнено полностью&выполнено частично&не выполнено\\
\hline
1&-&-&-&-\\
\hline
2&-&-&-&-\\
\hline
3&-&-&-&-\\
\hline
\end{tabular}

\subsection{Шаг 3}
Объедините ячейки\\
\begin{tabular}{|c|c|c|c|c|}
\hline
№&Критерии&\multicolumn{3}{|c|}{Параметры}\\
\hline
-&-&выполнено полностью&выполнено частично&не выполнено\\
\hline
1&-&-&-&-\\
\hline
2&-&-&-&-\\
\hline
3&-&-&-&-\\
\hline
\end{tabular}
\end{document} % конец документа